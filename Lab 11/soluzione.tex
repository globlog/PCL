\documentclass[a4paper]{article}
\usepackage{amsmath}
\begin{document}
\phantom{.}\\
L'esercizio equivale a risolvere l'equazione
 $$x\,\textvisiblespace\textvisiblespace\textvisiblespace\textvisiblespace\textvisiblespace\cdot \frac{a}{b}=\textvisiblespace\textvisiblespace\textvisiblespace\textvisiblespace\textvisiblespace\, x\iff x\,\textvisiblespace\textvisiblespace\textvisiblespace\textvisiblespace\textvisiblespace\cdot a=\textvisiblespace\textvisiblespace\textvisiblespace\textvisiblespace\textvisiblespace\, x\cdot b$$
Ora ponendo $\textvisiblespace\textvisiblespace\textvisiblespace\textvisiblespace\textvisiblespace=y$ di $n$ cifre otteniamo
$$x\,\textvisiblespace\textvisiblespace\textvisiblespace\textvisiblespace\textvisiblespace=x\cdot 10^n+y\qquad y<10^n$$
e allo stessso modo
$$\textvisiblespace\textvisiblespace\textvisiblespace\textvisiblespace\textvisiblespace\, x=10\cdot y+x$$
così l'equazione iniziale diventa
$$(x10^n+y)\cdot a=(10y+x)\cdot b\iff y=\frac{x10^na-bx}{10b-a}$$
esploro ora lo spazio delle $n\in\{0,\dots,8\}$ e $x\in\{1,\dots,9\}$ ed è risolto.
\end{document}